\documentclass[a4paper]{article}

\usepackage[english]{babel}
\usepackage{enumerate}
\usepackage{listings}
\usepackage{graphicx}

\lstset{frame=tb,
  language=Java,
  aboveskip=3mm,
  belowskip=3mm,
  showstringspaces=false,
  columns=flexible,
  basicstyle={\small\ttfamily},
  numbers=none,
  numberstyle=\tiny,
  breaklines=true,
  breakatwhitespace=true
  tabsize=2
}


\setlength{\topmargin}{0in}

\title{The Typo Problem}


\begin{document}
\begin{center}
Dylan Forbes\\
The Typo Problem
\end{center}
Problem: What is the asymptotic complexity of the number of times the following code will print with respect to $n$?
\begin{lstlisting}[frame=single]
for (int i=0; i < n; i++) {
	for (int j=0; j < i; j++) {
		if (j & i == 0) {
			System.out.println(i + " " + j);
		}
	}
 }
\end{lstlisting}

There are essentially two steps in finding the solution. First, derive a formula $N(i)$ for how many natural numbers $j$ below a given $i$ have a binary representation that has a 0 in every place where the binary representation of $i$ has a 1 (i.e. $j \& i == 0$). Then, use that formula to derive a formula $T(n)$ for the sum of $N(i)$ where $i$ spans 0 to $n$.
\\\\To begin, it is helpful to make a table of all relevant values:

\begin{center}
\begin{tabular}{| c | l | l | c | c |}
\hline
$n,i$&base2$(n)$&$J(i)=\{j:0\leq j\leq i \land j\&i=0\}$&$N(n)=|J(n)|$&$T(n)=\Sigma_{i=0}^n N(i)$\\
\hline
0&0& &0&0\\
1&1&0&1&1\\
2&10&0,1&2&3\\
3&11&0&1&4\\
4&100&0,1,10,11&4&8\\
5&101&0,10&2&10\\
6&110&0,1&2&12\\
7&111&0&1&13\\
8&1000&0,1,10,11,100,101,110,111&8&21\\
9&1001&0,10,100,110&4&25\\
10&1010&0,1,100,101&4&29\\
11&1011&0,100&2&31\\
12&1100&0,1,10,11&4&35\\
13&1101&0,10&2&37\\
14&1110&0,1&2&39\\
15&1111&0&1&40\\
16&10000&(16 items)&16&56\\
\vdots&\vdots&\vdots&\vdots&\vdots\\
\end{tabular}
\end{center}
\\\\
\indent Here, $J(i)$ is the set of integers $j$ for which the program will print for a given $i$; $N(i)$ is the number of items in $J(i)$, and thus the number of times the program will print within one iteration of the outer loop (with $j$ spanning 0 to $i$). Finally, $T(n)$ is the sum of all values of $N(i)$ where $i<n$, and is thus the number of times the program will print for the given value of $n$.

\\To figure out an expression for $N(i)$, first notice that there is a definite pattern in its values:
	\\\\
	\begin{tabular}{ c || c | c c | c c c c | c c c c c c c c c }
	$i$&1&2& &4& & & &8& & & & & & & &\cdots\\
	$N(i)$&1&2&1&4&2&2&1&8&4&4&2&4&2&2&1&\cdots
	\end{tabular}
	\\\\
	It seems to be a recursive pattern, where each successive block of values of $N(i)$ is composed of a copy of the previous block with its values doubled, followed by a regular copy. That is, if we let $I(x)$ refer to the $2^x$-tuple of the values of $N(i)$ where $2^x \leq i \leq 2^{x+1}-1$, then:
\begin{displaymath}
I(x)=\left\{
\begin{array}{lr}
$Append$(\{2i|i\in I(x-1)\},I(x-1))&:~x>0\\
\{1\}&:~x=0\\
\end{array}
\end{displaymath}
Where Append$(A,B)$ gives the ordered $(|A|+|B|)$-tuple of the values of $A$ followed by the values of $B$ with their orders preserved.
\\\\We can use this fact to determine a formula for $N(i)$, by summing up the chunks of $N(i)$ that precede $i$.
\\\\Let $S(x)=\Sigma (I(x))$. ($x$ counts $2^x$-long blocks of values, so it is essentially $\lfloor \log_2 n\rfloor$ for a given $n$.) \\Then according to the above formula for $I(x)$,
\begin{displaymath}
S(x)=\left\{
\begin{array}{lr}
\Sigma\{2i|i\in I(x-1)\}+\Sigma I(x-1))&:~x>0\\
\Sigma\{1\}&:~x=0\\
\end{array}
\end{displaymath}
\\
\begin{displaymath}
=\left\{
\begin{array}{lr}
2\Sigma I(x-1)+\Sigma I(x-1))&:~x>0\\
\Sigma\{1\}&:~x=0\\
\end{array}
\end{displaymath}
\\
\begin{displaymath}
=\left\{
\begin{array}{lr}
3\Sigma I(x-1)&:~x>0\\
\Sigma\{1\}&:~x=0\\
\end{array}
\end{displaymath}
\\
\begin{displaymath}
=\left\{
\begin{array}{lr}
3S(x-1)&:~x>0\\
1&:~x=0\\
\end{array}
\end{displaymath}
Thus, a closed formula for $S(x)$ is
	$$S(x)=3^x$$
And indeed, the table reflects this:
\begin{center}
\begin{tabular}{ c | r | c | l }
$x$&$I(x)$&$S(x)$&$[2^x,2^{x+1}-1]$~(spanned values of $n$)\\
\hline
0&\{1\}&1&$[1]$\\
1&\{2,1\}&3&$[2,3]$\\
2&\{4,2,2,1\}&9&$[4,7]$\\
3&\{8,4,4,2,4,2,2,1\}&27&$[8,15]$\\
\vdots&\vdots&\vdots\\
\end{tabular}
\end{center}
Now, note that if $n=2^x$ for some integer $x>0$, then $S(x-1)$ gives the sum of the $2^x$-long ``block'' of entries in $N$ that precedes $N(n)$'s block\textemdash that is, $S(x-1)=\Sigma_{x=n/2}^{n-1}(N(x))$.\\
So, because $T(n)$ is defined as the sum of all entries $N(x)$ where $0\leq x \leq n$, $T(n-1)$ must be the sum of all ``blocks'' that precede $n$'s block (since the blocks are contiguous). And since $n$ is the first entry in its block, because it is assumed to be a power of 2,
$$T(n-1)=\Sigma_{i=0}^{n-1} N(i)$$
Then, because $N(n)=n$ for $n=2^x$ (this can be seen from the original table),
$$T(n)=T(n-1)+N(n)=T(n-1)+n$$
Thus we have the following formula for $T$:
$$T(n)=n+\Sigma_{x=0}^{\log_2(n)-1}(S(x))$$
So, combining this with the fact that $S(x)=3^x$, we have
$$T(n)=n+\Sigma_{x=0}^{\log_2(n)-1}(3^x)$$
which simplifies to
$$T(n)=n+\frac{1}{2}(3^{\log_2(n)}-1)$$
all for $n=2^x$.

To see that this is correct, construct another table of values, where $T(n)$ is the value generated iteratively from the table, and $T?(n)$ is the value given by the previously derived formula for $T(n)$ (which we only showed works for $n=2^x$):
\begin{center}
\begin{tabular}{ c  | c | c }
$n$ & $T(n)$ & $T?(n)$\\
\hline
0&0&$-0.5$\\
1&1&1\\
2&3&3\\
3&4&5.35\\
4&8&8\\
5&10&10\\
6&12&14.06\\
7&13&17.42\\
8&21&21\\
9&25&24.77\\
10&29&28.73\\
11&31&32.86\\
12&35&37.17\\
13&37&41.64\\
14&39&46.27\\
15&40&51.06\\
16&56&56\\
\vdots&\vdots&\vdots\\
\end{tabular}
\end{center}

Although $T?(n)=T(n)$ only for $n=2^x$, both functions must have the same asymptotic complexity, because $T?(n)\geq T(n)$ for all $n>0$, but $T?(n)$ never gets too far ahead of $T(n)$ since they are equal for powers of 2.
\\Thus, $T(n)\in O(n+\frac{1}{2}(3^{\log_2(n)}-1))$.

Then, finally, because

$$\lim_{n\to\infty}\frac{n+\frac{1}{2}(3^{\log_2(n)}-1)}{3^{\log_2(n)}}=\lim_{n\to\infty}\frac{3^{\log_2(n)}}{3^{\log_2(n)}}=1$$

exists and is finite, $T(n)\in O(3^{\log_2 n})$, which is between and $O(n \log n)$ and $O(n^2)$.\\


\\\\So the answer is $O(3^{\log_2 n})$.

\end{document}
